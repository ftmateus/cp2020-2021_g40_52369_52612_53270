\IEEEPARstart{T}{o} date, computers have not only increased in raw processing power, but, more importantly, in the number of processing cores in their central processing units (CPU). This substantially increases the number of tasks that can be done at the same time, something a single core would have difficulty doing given the physical constraints that prevent further frequency scaling. This type of computing is known as parallel computing and it is important for the programmer to learn some of the related concepts to make better use of the hardware. This project intends to apply some concepts taught in Concurrency and Parallelism classes, such as parallel patterns and data dependencies, in a program that simulates the effects of high energy particle bombardment on an exposed surface, for example on the surface of spacecraft in outer space. The development of the paralleled version of this program was aided with the use of the OpenMP (Open Multi-Processing) application programming interface (API), a very handy tool that simplifies multi threading program development since the programmer can start developing from a sequential version of a program without usually much changes in the source code.

