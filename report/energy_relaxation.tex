\subsubsection{Description}

In this phase the material reacts by slightly distributing its charge. Each cell is updated with the average value of three cells: the previous cell, the cell itself, and the next cell. The extremes of the layer are not updated during this phase.

\subsubsection{Sequential version analysis}
To avoid destroying the original values in the array while they are still needed to update other neighbors, the array values are first copied to a second ancillary array before the relaxation. The old values are read from the ancillary array, while the new values are written to the original array.
There is both a true dependency and an anti-dependency in this process.

\subsubsection{Parallel version implementation}

It was evident that creating the ancillary layer to preserve the old values of the original one creates a lot of overhead each time a storm file is processed, so it was natural to think about removing this array, not only to improve memory consumption, but also speedup. 

However, doing so implies resolving some complex loop-carried dependencies, besides preserving the old values in some other way, like using ancillary variables.

First, instead of a whole ancillary array, there would only be needed 2 values: the original value of the previous cell and the original value of the current cell. After updating the current cell, its previous cell's original value becomes the current cell's original value, and the cycle proceeds to updating the next cell.

Second, in order to make this process parallelizable, the layer would be split equally by the threads. However, since each cell needs its neighbors, the extremes of each sub-layer would need to be given ancillary cells with the original values of their missing neighbors. This process is done before the updating process. This is a implementation of a Map related pattern known as \textbf{stencil}, where each instance of the map function accesses neighbors of its input, offset from its usual input.

\par 